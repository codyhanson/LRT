\section{Proposed design}
\label{section:proposeddesign}
One of the key goals of any instrumentation inserted into an application is 
that it should not adversely affect the performance or behavior of the 
application execution. With this fact in mind, we realize that it will be 
impossible to introduce merely negligable overhead while still gathering
meaningful amounts of data. Still, we will seek to make data collection
and exportation as minimally impactful on phone energy, bandwidth and processing
resources as possible.

\subsection{Server design}
One of the key pieces of our system is the LRT Server, which serves
to collect trace data from mobile phones via a network connection.
One server should be able to handle many phone clients exporting data at once.
The server will be responsible for adding data that it recieves into
some datastore that it has access to. We will initially do minimal realtime
processing of the data, but could expand this in the future to improve the
analytical power of the service. The first priority is for the server to quickly
and efficiently persist the data that it recieves.

In light of these requirements, we think the Node.js \cite{node}
project to be a good fit for our server platform.
Node.js is an asynchronous, event driven server 
which interprets JavaScript with the V8 JavaScript engine \cite{V8-javascript} that
powers the Google Chrome browser. Node.js is especially good
at maintaining many concurrent connections in a single threaded implementation.
Because it doesn't block on IO operations, we think it will be an efficient broker
between the client mobile phones and the datastore. but will use a different server
system to perform analytics
because any heavy processing by a request can tie up Node's event loop.
Due to Node's rich community and JavaScript platform,
we will be able to prototype the server software with minimal development friction.

\subsection{Efficient network transport}
In order to minimize the overhead of exporting trace 
events to the server, there are two possible schools of networking thought.
One option is the relatively new Websocket protocol 
\cite{WebsocketRFC}, which keeps open a bidirectional communications channel 
over the HTTP port. By keeping open a long lived 
websocket connection and sending size efficient 
messages, we do not suffer the TCP and HTTP connection establishment overhead 
for each trace event. Websockets have an advantage over a custom TCP protocol
in that because it looks like HTTP traffic, it is able to pass through most
firewalls. It is also easier on the server side to implement a websocket
connection using Node.js than a traditional TCP stream.

Our study of \cite{PeriodicTransfers} suggests that
our instrumentation should follow a more sparse communications pattern, making
a RESTful API\cite{REST} design favorable over websockets. If communications
happen infrequently in order to save resources, the new connection overhead for
each HTTP request will affect resources less than the battery
drain a continuous HTTP connection of Websockets would incur.
Each HTTP POST could send an array of compressed trace events, making
efficient use of the cellular or wifi radio. Another approach could be to 
devise a way to detect when the radio is already awake, and make a transmission
opportunistically. REST also has the advantage of being the de-facto communications
scheme for cloud services that use HTTP. Implementing a RESTful API using Node.js is trivial.

If it seems that accessing a cloud server is too expensive, this same technique could
be applied to a server on the LAN. A local server would have reduced latency, and higher network throughput,
compared to a cloud resource. However, if the cloud access penalty is acceptable, we believe this provides
a superior use model that would make people more likely to use the technology. 

\subsection{Datastore considerations}
Thinking ahead to realistic user volumes for a service of this nature, it is
important to think about the model for storing data that best fits the 
needs of our design. We will need to be able to store millions of trace points
that span data collected from many different users. User's privacy must also be
considered, to ensure that sensitive and proprietary information is not shared
inadvertently with other users. For our prototype we will use MongoDB \cite{MongoDB}, a 
JSON \cite{JSON} document oriented NoSQL datastore, known for its scalability and rich
querying functionality. With MongoDB, it is possible to run MapReduce \cite{MapReduce}
queries over the data, which could be useful for data analytics collection.
MongoDB is also very easy to use, which will allow us to rapidly prototype.
We believe this to be a good initial choice over alternatives like PostgreSQL \cite{Postgres}
due to MongoDB's lack of schema, and ability to store large collections of JSON documents. 
In order to take advantage of MongoDB's programming model, we will represent 
each trace and trace event as a JSON document.

\subsection{Instrumentation for the Android platform}
In order to capture trace events generated by an application, we will build
a service app which runs continually in the background of the mobile phone.
Android has a class called \texttt{BroadcastReciever} which can receive messages
from multiple processes at the same time. An instrumented process merely needs
to know the right \texttt{Intent} to broadcast with, and arbitrary data can be
sent to the service. Code inserted into the App under test will communicate with the
background service via static methods in a Java class we will create.
The service will recieve events from applications under test, buffering them until the
next appropriate time to export them to the server. This allows us to support 
tracing for any and all apps on the device so long as our service is running and 
that app has been injected with broadcasting code.

\subsection{Enabling ease of use with automatic instrumentation}
If time permits, we would like to build an automated instrumentation insertion 
script, which would detect the appropriate spots in the App code to insert our 
code. This would make it easy for any developer to apply the LRT capability to
an existing app, just by running the script and making a debug build of their 
app. Possible insertion points of interest could be program structures such as the first
line of each method, or within \texttt{if/else} blocks, or within the body of
a loop structure. The script would also populate the trace calls with relevant data
for you, such as line number, class, method, and perhaps even basic state info (battery
level, CPU usage, etc...).


