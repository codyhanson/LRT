\section{Related Work}
Existing techniques for tracing program execution are decades old, and usually 
involve writing lines of output that the programmer purposefully incorporates 
in the code into a file on disk, or exported to a logging server, as in syslog. 
Disk output can slow down a program, and the amount of log information that 
builds up could be too large for the mobile device to store. The work done in 
\cite{NSLogger} has the remote debugging capability for the iOS platform, although 
it does not have the analytics that we will propose, and it requires user action 
to implement the tracing. Android has remote debugging built into the developer 
SDK \cite{AndroidRemoteDebugTool}, but this requires your app to be tethered to a development PC.

Several businesses have been formed to address the need in this space, including
Flurry \cite{Flurry} which provides app developers with high level metrics on
user behavior and app performance, and Heatmaps \cite{Heatmaps}, an iOS specific
service which gives developers deep insight to how users interact with a touch
screen, and how they transition through the screens of an app. 
Crashlytics \cite{Crashlytics} is a popular service that exports detailed
crash analysis data from user devices to the cloud. There is a 'live' component,
to the service, but it is focused on presenting crash statistics as they happen,
not program step-by-step trace data.

There has also been pertinent work done in more academic circles. If we can 
generalize the work from other fields to apply to LRT, we can save ourselves 
time and boost the quality of our research. In \cite{ProfileDroid}, a 
multi-layer analysis tool for Android apps is presented, which employs a 
technique to track user input via the Android tools 'getevent' and 'logcat'. 
This work also collects system call invocation via the 'strace' tool. 

We can also draw ideas from the taint tracing techniques discussed in 
\cite{TaintDroid} as a way to implement our trace hooks while minimizing 
the overhead of our tool. While we could have leveraged their method of 
tracing by developing near the kernel, we opted for the convenience and 
speed of development using built-in Android tools like Broadcasts to prove
out the idea more quickly. The techniques discussed in \cite{PeriodicTransfers}
will be a powerful resource for implementing our event reporting service 
with energy and network resource consumption.


