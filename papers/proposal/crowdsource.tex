
\section{Crowd sourcing for large scale trace acquisition}
One of the key issues with testing mobile applications is the sheer number
of different scenarios that can affect the execution and performance
of the App under test. It is impossible for developers to test their application
on every hardware configuration, for all use models, for all environmental scenarios.
Due to this fact, we believe that using the LRT system to 'crowd source' data for developers
is the answer.

\section{Motivating Participation}
Any savvy user will understand that opting into using this system comes at a cost to themselves.
Their phone will be expending extra energy and network capacity to support the tracing of 
other peoples applications, which they build for fun and profit.
In order to encourage more users to opt in, we will need to compensate them for their
help, or make it fun for them to volunteer.

\subsection{Gameification}
A common way to motivate user engagement is with gameification. This is typically embodied
by trophies, achievements, and leaderboards. App testers would earn trophies or achievements
for successful testing, in the hopes that they are encouraged to continue testing, and at an
increased rate. Additionally, having leaderboards encourages them to bring in friends and colleagues
to compete with.

A more interesting and, perhaps, novel implementation of gameification would be to reward developers
for high code quality. We might flag our trace events with a severity (Info, Warning, Error), allowing
us to rate an app by how few errors it has. An additional gameification metric to score developers on
could be on "testing confidence," or the variety of device variations their app has been tested on.
This would encourage the use of higher tiers of testing service.

\subsection{A Market Based Approach}
One of the best ways to reward participation is with money. As the relayer between app developers
and app testers, we could charge the developers to have their apps tested. This might be priced
as a metered, tiered approach - using more testers costs more, and getting results quicker costs more.
On the tester side, we could reward users in a number of ways. One would be to simply put a
value on each trace event - each trace event they produce is worth \$0.xx. We could improve the
effectiveness of our "TaaS marketplace" by making the value of trace events dynamic. Producing a trace
event on a high value app (where the developer is at a high tier or is paying for quicker results) would
be worth more than a low value app. Additionally we could draw from our Gameification
ideas and place a higher value on less frequently produced trace events. In all likelihood, these
less frequent events would be bugs; therefore finding bugs and boundary cases is worth more to the 
testers than following "hot paths". By encouraging testers to search for bugs and edge cases, we
provide a higher value to the developers.

