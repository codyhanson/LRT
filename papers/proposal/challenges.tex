\section{Challenges}
\label{section:challenges}
Due to the ambitious technical scope of this project, there are a number of challenges we must overcome.
In this section, we overview the main challenges. Section \ref{section:proposeddesign} discusses
our solutions to these challenges.

\subsection{Instrumentation}
We must find a way to insert our instrumentation into mobile apps that will
allow for our remote tracing to occur. Automated insertion would be preferable,
but to expedite prototyping, we have opted for manual insertion initially.
While there has been some of research about efficiently and conveniently inserting
instrumentation into mobile apps, much of it targets a lower-level implementation than
we hope to achieve, such as coding at the kernel level to trace taints \cite{TaintDroid}.
There is also the Aspect Oriented Programming approach \cite{COCA}, as mentioned in Related Work.

\subsection{Reporting Trace Data to the Server}
We must implement our data reports to the backend server in such a way that it
has a minimal impact on mobile device performance and battery drain. The study done
in \cite{PeriodicTransfers} indicates we should accumulate trace data for some
period of time, then send it in bursts, to make best use of the energy properties
of the cellular radio. We will need to experiment with various
periods between transmissions to find a balance between keeping the accumulated data to a manageable
size while being cognizant of the cellular radio's requirements in order to minimize
energy usage.

\subsection{Trace Data Storage in the Cloud}
Scaling the storage of data in cloud services has never been easier,
but it is still a problem which warrants some careful thought and design.
Our service would have to handle potentially hundreds of thousands of trace's,
comprised of millions of trace events. We also need a datastore that will allow
us to efficiently and flexibly query the data to find properties of the executions that will
provide insight to developers.

\subsection{Analytics}
Once we have the trace data, we need to do something with it to inform developers
what action they can take to improve the user's experience. To break this task
into a realistic size, we will initially do some basic data roll-up, grouping
traces by type and comparing the average time between traces. A stretch goal
would be to use a real-time analytics tool or machine learning system 
to make it easier for the developer to find actionable tasks. With 
an opt in system to protect privacy, we could compare trace data between users,
to expose statistical data such as average time to load a view, or average 
application startup time. Knowing that baseline, a developer could know 
if maybe they need to improve their application in these areas.

% clh: This section seems a bit unfocused and out of place, compared to the rest of the writing.
%\subsection{Application Adaptation}
%The final piece that would make this tool truly unique would be to take the output
%of our analytics \ref{Analytics} to drive the application to change the UI to
%another layout that would provide a better user experience. The major difficulty will
%be finding a reasonable way to change the UI that is not over complicated, but will
%have a positive impact on the UI. Additionally, telling the application what to change
%and how will be challenging. Something we can do to simplify this is to constrain what
%we are willing to change to list-based views, as lists are common and can be displayed
%in a number of ways, such as scrolling listviews, grids, or paged lists.

\subsection{Participation}
Our final challenge will be encouraging users to test these LRT-enabled apps. Our
proposal relies on crowdsourcing to provide the data we need to improve user
experience by testing with real people. Since we want regular people, not employees
of the app, to do the testing, we will need to provide incentive. Some possibilities
for incentivizing the use of our tool are discussed in \ref{section:crowdsourcing}. They include
a market based approach (let users earn money by opting in to testing), and gameification.


