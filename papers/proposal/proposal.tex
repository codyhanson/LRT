\documentclass{acm_proc_article-sp}

\usepackage{graphicx}


\usepackage{url}
\usepackage{hyperref} 
\hypersetup{breaklinks} 

\begin{document}
\title{LRT: The Live Remote Trace service for Mobile Debugging}
\author{Cody Hanson, Ross Nordstrom\\
        University of Colorado - Colorado Springs\\
        \texttt{\{chanson,rnordstr\}@uccs.edu}
       }
\date{June 2013}

\maketitle

\begin{abstract}
A concise and informative abstract will live here one day.
\end{abstract}

\section{Introduction}
One of the key tasks a mobile application developer must undertake is testing 
their application on real hardware, and with real users. This is important 
because there are many features that are available to apps running on hardware 
that are not available in emulation software. These include GPS, accelerometer, 
magnetometer, gyroscope, and Cellular data, as well as the rest of the 
complexity of the system that is impossible to recreate perfectly in an 
emulation layer. Real users and their environments also may exercise apps in
a way developers have not anticipated. 

We propose a system which will allow developers of applications on internet connected 
devices to easily embed lightweight instrumentation 
into their application code to enable high resolution 
tracing of program execution to a cloud service. Events are collected by 
the service which can perform analytics on program execution as it is 
exercised out in the wild.

Our research plan consists of prototyping a vertical slice of the system.
This includes the server infrastructure which will recieve trace data,
a background service which runs on the mobile platform, and a sample
application which has instrumentation that can broadcast to the background service.
In addition to investigating technologies that would make this technique 
feasible, we also will attempt to observe the processing and data workloads
involved, to better be able to select technologies used in a production version.

Successful completion of this work will serve to accomplish a few key 
improvements for testing mobile applications. This instrumentation will provide 
the developer with a trace of the program running on actual hardware, not just 
an emulator, as the opportunity to collect the trace while the application is 
subjected to real life users and environment interaction. With this 
body of program flow data collected on the server, we hypothesize that we can mine it for useful 
information to answer questions such as ``\emph{which parts of the user interface were 
confusing to the user?}'', and ``\emph{Did they spend too much time wondering what to do 
next?}'', or ``\emph{which flows are common for users to take, and could thus be 
streamlined to create a better user experience?}'' Problems of this nature are 
discussed in \cite{WebAntiPattern}.

In order for this work to be successful, we will need to 
prove out a few technical approaches. The first one will be to determine at what rate can we
export trace events from the application under test to the background LRT service 
before the user experience begins to degrade. 
We will also need to find out on what order of magnitude is the data
that we are collecting, and how that will impact the export-to-server performance.
Finally, we will attempt to write some simple queries against our database of
trace events, to gain some initial statistics and data that would give a developer
actionable information about the behavior of their application.

The remainder of this paper is organized as follows. Up front, we will cover 
related work in section \ref{section:relatedwork}. In section \ref{section:challenges},
we will discuss considerations that must be made when designing our system, and in 
\ref{section:proposeddesign} we will address these challenges with an outline of
our proposed design. In section \ref{section:crowdsourcing} we will discuss how to drive
user participation for crowdsourcing, which is a critical non-technical 
challenge to make our approach successful. The techniques to mine data for useful 
information are outlined in \ref{section:datamining}
Finally, we will discuss how we plan to evaluate our work in \ref{section:evaluation}.

\section{Related Work}
Existing techniques for tracing program execution are decades old, and usually 
involve writing lines of output that the programmer purposefully incorporates 
in the code into a file on disk, or exported to a logging server, as in syslog. 
Disk output can slow down a program, and the amount of log information that 
builds up could be too large for the mobile device to store. The work done in 
\cite{NSLogger} has the remote debugging capability for the iOS platform, although 
it does not have the analytics that we will propose, and it requires user action 
to implement the tracing. Android has remote debugging built into the developer 
SDK \cite{AndroidRemoteDebugTool}, but this requires your app to be tethered to a development PC.

Several businesses have been formed to address the need in this space, including
Flurry \cite{Flurry} which provides app developers with high level metrics on
user behavior and app performance, and Heatmaps \cite{Heatmaps}, an iOS specific
service which gives developers deep insight to how users interact with a touch
screen, and how they transition through the screens of an app. 
Crashlytics \cite{Crashlytics} is a popular service that exports detailed
crash analysis data from user devices to the cloud. There is a 'live' component,
to the service, but it is focused on presenting crash statistics as they happen,
not program step-by-step trace data.

There has also been pertinent work done in more academic circles. If we can 
generalize the work from other fields to apply to LRT, we can save ourselves 
time and boost the quality of our research. In \cite{ProfileDroid}, a 
multi-layer analysis tool for Android apps is presented, which employs a 
technique to track user input via the Android tools \emph{getevent} and \emph{logcat}. 
This work also collects system call invocation via the \emph{strace} tool. 

We can also draw ideas from the taint tracing techniques discussed in 
\cite{TaintDroid} as a way to implement our trace hooks while minimizing 
the overhead of our tool. While we could have leveraged their method of 
tracing by developing near the kernel, we opted for the convenience and 
speed of development using built-in Android tools like Broadcasts to prove
out the idea more quickly. The techniques discussed in \cite{PeriodicTransfers}
will be a powerful resource for implementing our event reporting service 
with minimized energy and network resource consumption.

TODO: cite COCA: Computation Offload to Clouds Using AOP

TODO: cite JaBUTi/ME


\section{Challenges}
\label{section:challenges}
Due to the ambitious technical scope of this project, there are a number of challenges we must overcome.
In this section, we overview the main challenges. Section \ref{section:proposeddesign} discusses
our solutions to these challenges.

\subsection{Instrumentation}
We must find a way to insert our instrumentation into mobile apps that will
allow for our remote tracing to occur. Automated insertion would be preferable,
but to expedite prototyping, we have opted for manual insertion initially.
While there has been some of research about efficiently and conveniently inserting
instrumentation into mobile apps, much of it targets a lower-level implementation than
we hope to achieve, such as coding at the kernel level to trace taints \cite{TaintDroid}.
There is also the Aspect Oriented Programming approach \cite{COCA}, as mentioned in Related Work.

\subsection{Reporting Trace Data to the Server}
We must implement our data reports to the backend server in such a way that it
has a minimal impact on mobile device performance and battery drain. The study done
in \cite{PeriodicTransfers} indicates we should accumulate trace data for some
period of time, then send it in bursts, to make best use of the energy properties
of the cellular radio. We will need to experiment with various
periods between transmissions to find a balance between keeping the accumulated data to a manageable
size while being cognizant of the cellular radio's requirements in order to minimize
energy usage.

\subsection{Trace Data Storage in the Cloud}
Scaling the storage of data in cloud services has never been easier,
but it is still a problem which warrants some careful thought and design.
Our service would have to handle potentially hundreds of thousands of trace's,
comprised of millions of trace events. We also need a datastore that will allow
us to efficiently and flexibly query the data to find properties of the executions that will
provide insight to developers.

\subsection{Analytics}
Once we have the trace data, we need to do something with it to inform developers
what action they can take to improve the user's experience. To break this task
into a realistic size, we will initially do some basic data roll-up, grouping
traces by type and comparing the average time between traces. A stretch goal
would be to use a real-time analytics tool or machine learning system 
to make it easier for the developer to find actionable tasks. With 
an opt in system to protect privacy, we could compare trace data between users,
to expose statistical data such as average time to load a view, or average 
application startup time. Knowing that baseline, a developer could know 
if maybe they need to improve their application in these areas.

% clh: This section seems a bit unfocused and out of place, compared to the rest of the writing.
%\subsection{Application Adaptation}
%The final piece that would make this tool truly unique would be to take the output
%of our analytics \ref{Analytics} to drive the application to change the UI to
%another layout that would provide a better user experience. The major difficulty will
%be finding a reasonable way to change the UI that is not over complicated, but will
%have a positive impact on the UI. Additionally, telling the application what to change
%and how will be challenging. Something we can do to simplify this is to constrain what
%we are willing to change to list-based views, as lists are common and can be displayed
%in a number of ways, such as scrolling listviews, grids, or paged lists.

\subsection{Participation}
Our final challenge will be encouraging users to test these LRT-enabled apps. Our
proposal relies on crowdsourcing to provide the data we need to improve user
experience by testing with real people. Since we want regular people, not employees
of the app, to do the testing, we will need to provide incentive. Some possibilities
for incentivizing the use of our tool are discussed in \ref{section:crowdsourcing}. They include
a market based approach (let users earn money by opting in to testing), and gameification.



\section{Proposed design}
\label{section:proposeddesign}
One of the key goals of any instrumentation inserted into an application is 
that it should not adversely affect the performance or behavior of the 
application execution. With this fact in mind, we realize that it will be 
impossible to introduce merely negligable overhead while still gathering
meaningful amounts of data. Still, we will seek to make data collection
and exportation as minimally impactful on phone energy, bandwidth and processing
constraints as possible.

\subsection{Server design}
One of the key pieces of our system is the LRT Server, which serves
to collect trace data from mobile phones via a network connection.
One server should be able to handle many phone clients exporting data at once.
The server will be responsible for adding data that it recieves into
some datastore that it has access to. We will initially do minimal realtime
processing of the data, but could expand this in the future to improve the
analytical power of the service. The first priority is for the server to quickly
and efficiently persist the data that it recieves.

In light of these requirements, we think the Node.js \cite{node}
project to be a good fit for our server platform.
Node.js is an asynchronous, event driven server 
which interprets JavaScript with the V8 JavaScript engine \cite{V8-javascript} that
powers the Google Chrome browser. Node.js is especially good
at maintaining many concurrent connections in a single threaded implementation.
Because it doesn't block on IO operations, we think it will be an efficient broker
between the client mobile phones and the datastore. but will use a different server
system to perform analytics
because any heavy processing by a request can tie up Node's event loop.
Due to Node's rich community and JavaScript platform,
we will be able to prototype the server software with minimal development friction.

\subsection{Efficient network transport}
In order to minimize the overhead of exporting trace 
events to the server, there are two possible schools of networking thought.
One option is the relatively new Websocket protocol 
\cite{WebsocketRFC}, which keeps open a bidirectional communications channel 
over the HTTP port. By keeping open a long lived 
websocket connection and sending size efficient 
messages, we do not suffer the TCP and HTTP connection establishment overhead 
for each trace event. 

Event messages can also be compressed and buffered for when a 
network connection is again available. We will use a websocket enabled HTTP 
server, and establish feasibility of running the server on a hosted cloud 
provider, or on the LAN if performance dictates. If the cloud access penalty is 
acceptable, we believe this could be a superior use model that would make people
more likely to use the technology. 

However our study of \cite{PeriodicTransfers} suggests that
our instrumentation should follow a more sparse communications pattern, making
a RESTful API\cite{REST} design favorable over websockets. If communications
only happen infrequently in order to save resources, the new connection overhead for
each HTTP request will affect resources less than the battery
drain a continuous HTTP connection of Websockets would incur.

\subsection{Datastore considerations}
Thinking ahead to realistic user volumes for a service of this nature, it is
important to think about the model for storing data that best fits the 
needs of our design. We will need to be able to store millions of trace points
that span data collected from many different users. User's privacy must also be
considered, to ensure that sensitive and proprietary information is not shared
inadvertently with other users.

\subsection{Instrumentation for the Android platform}
In order to capture trace events generated by an application, we will build
a service app which runs continually in the background of the mobile phone.
Android has a class called \texttt{BroadcastReciever} which can receive messages
from multiple processes at the same time. An instrumented process merely needs
to know the right \texttt{Intent} to broadcast with, and arbitrary data can be
sent to the service. Code inserted into the App under test will communicate with the
background service via static methods in a Java class we will create.
The service will recieve events from applications under test, buffering them until the
next appropriate time to export them to the server. This allows us to support 
tracing for any and all apps on the device so long as our service is running and 
that app has been injected with broadcasting code.

\subsection{Enabling ease of use with automatic instrumentation}
If time permits, we would like to build an automated instrumentation insertion 
script, which would detect the appropriate spots in the App code to insert our 
code. This would make it easy for any developer to apply the LRT capability to
an existing app, just by running the script and making a debug build of their 
app. Possible insertion points of interest could be program structures such as the first
line of each method, or within \texttt{if/else} blocks, or within the body of
a loop structure. The script would also populate the trace calls with relevant data
for you, such as line number, class, method, and perhaps even basic state info (battery
level, CPU usage, etc...).




\section{Crowd sourcing for large scale trace acquisition}
One of the key issues with testing mobile applications is the sheer number
of different scenarios that can affect the execution and performance
of the App under test. It is impossible for developers to test their application
on every hardware configuration, for all use models, for all environmental scenarios.
Due to this fact, we believe that using the LRT system to 'crowd source' data for developers
is the answer.

\section{Motivating Participation}
Any savvy user will understand that opting into using this system comes at a cost to themselves.
Their phone will be expending extra energy and network capacity to support the tracing of 
other peoples applications, which they build for fun and profit.
In order to encourage more users to opt in, we will need to compensate them for their
help, or make it fun for them to volunteer.

\subsection{A Market Based Approach}
talk about market

\subsection{Gameification}
talk about gameification



\section{Mining trace data}
Here describe what to do with the data

\section{Empirical Evaluation}
\label{section:evaluation}
This section describes what we did to evaluate our proposal, and how well it performed
based on metrics we predefined.

\subsection{Plan}
There a few main things that will justify our research as an improvement on existing techniques.

First, it must objectively improve user experience, and do so better than other techniques.
To measure this, we will compare the speed with which a user navigates through a new app
before and after our UI improvements. To control this, we will have users use the two variants
of the app in both orders: improved version after original, and improved version before original.
Additionally, we will investigate options for simulating user input.

Second, it would be good to measure the time it takes developers to achieve equivalent
improvements in user experience with alternative techniques. This will be difficult to
measure objectively. Something we could do is use a public, popular app and start with
an old version, then compare the improvements our tool suggests to those the app team
implemented subsequent to that version. If our tool leads to a similar or better improvement
than the app team implemented, and does it in a reasonable time, this would indicate the
value of the LRT system.

Third, we must show that the LRT tool does not unreasonably impact the device's performance.
Since LRT aims to test UX and trace remotely, we cannot have the same expectations as testing
tools that store the results on the device. Additionally, we can run experiments to see how
much data, and how detailed of information we trace, before performance degradation exceeds
some limit. We propose that this threshold should be 20\% worse battery consumption, and 10\%
CPU and memory overhead.

\subsection{Setup}
For ease of implementation, we developed LRT for Android. We will use a Samsung Galaxy S3
over Wi-Fi of two signal strength levels (60dBm and 100dBm), a 3G connection, and a 4G connection
(Verizon). We will select an open source app that relies on listviews to run our tests on. We will
do some initial experimentation on our own to judge what level of detail our instrumentation can
trace without exceeding our performance impact thresholds. Based on this, we will insert some
amount of instrumentation into the app and have 5 randomly selected UCCS students use the app
in a consistent cellular service environment. Based on the results of LRT, we will modify the
app to improve user experience, after which we'll have another 5 randomly selected UCCS students
use the modified app. To judge the success of LRT, we will compare the performance metrics of the
two groups of students, as well as a reaction survey they will take after participating in the
study.

\subsection{Experiment Outcomes}
...




\section{Conclusion}
...

\bibliography{citations}{}
\bibliographystyle{plain}

\end{document}

