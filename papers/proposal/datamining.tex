\section{Mining trace data}
\label{section:datamining}
Once we collect enough data, we can compare user experience between versions
of an app, as well as across apps by categorizing actions well. Additionally
we can compare trace events that occurred against a reference of what events
are possible to produce a rough idea of functional code coverage. Further
more, we can produce something along the lines of a "cross-platform coverage"
giving developers an indication of how well their app is tested on different
devices and OS'es. We could take this a step further to identify "troublesome"
devices and OS'es by looking for common error events across apps running on the
same device and OS.

\subsection{Time based profiling}
Since each trace event has a timestamp, we could mine the data for rich
temporal information, and apply statistics to let the developer know if their application
isn't performing as well as their competitors. One example could be that we have average 
view load times for many different applications, and if a particular app's load times
are consistantly bad, or have a very large standard deviation, we could flag this as a
point of concern for the developer to address.

\subsection{Button Mashing}
A great place to insert instrumentation will be in callbacks that respond to user input.
This will allow us to track how a user interacts with an application, and in what order
they interact. If we see a succession of quick button presses, maybe we can see that a user was 
frustrated. Or if we see a checkbox being quickly checked and unchecked, then we can tell that
the option is perhaps confusing, or causing people to spend a lot of time configuring it.
On the other end of the spectrum, if we see some statistics about how a majority of users
never use a particular UI control, it could perhaps be redefined or factored out since it might be
less valuable to users to have on the screen.
