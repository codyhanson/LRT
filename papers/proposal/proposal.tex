
\documentclass[11pt]{article}

\usepackage{graphicx}

\begin{document}
\title{LRT: The Live Remote Trace service for Mobile Debugging}
\author{Cody Hanson, Ross Nordstrom}
\date{June 2013}

\maketitle

\begin{abstract}
%\boldmath
balkasldkfhalsdkhfa;kdsf
balkasldkfhalsdkhfa;kdsf
balkasldkfhalsdkhfa;kdsf
balkasldkfhalsdkhfa;kdsf

\end{abstract}

\section{Introduction}
One of the key tasks a mobile application developer must undertake is testing 
their application on real live hardware, and with real users. This is important 
because there are many features that are available to apps running on hardware 
that are not available in emulation software. These include GPS, accelerometer, 
magnetometer, gyroscope, and Cellular data, as well as the rest of the 
complexity of the system that is impossible to recreate perfectly in an 
emulation layer. Real users and their environments also may exercise your app in
a way you have not anticipated. 

We propose a system which will allow an internet connected application to easily
embed lightweight instrumentation into their application code to enable real 
time tracing of program execution to a cloud service. Events are collected by 
the service which can perform analytics on your program execution as it is out 
in the wild.

Successful completion of this work will serve to accomplish a few key 
improvements for testing mobile applications. This instrumentation will provide 
the developer with a trace of the program running on actual hardware, not just 
an emulator, as well as real life user and environment interaction. With this 
body of program flow data collected on the server, we can mine it for useful 
information to answer questions such as “which parts of the user interface were 
confusing to the user?”, and “Did they spend too much time wondering what to do 
next?”, or “which flows are common for users to take, and could thus be 
streamlined to create a better user experience?”. Problems of this nature are 
discussed in [1].

\section{Related Work} 
Existing techniques for tracing program execution are decades old, and usually 
involve writing lines of output that the programmer purposefully incorporates 
in the code into a file on disk, or exported to a logging server, as in syslog. 
Disk output can slow down a program, and the amount of log information that 
builds up could be too large for the mobile device to store. The work done in 
[2] and [3] has the remote debugging capability for the iOS platform, although 
does not have the analytics that we will propose, and it requires user action 
to implement the tracing. Android has remote debugging built into the developer 
SDK [6], but this requires your app to be tethered to a development PC.

\section{Proposed design}
One of the key goals of any instrumentation inserted into an application is 
that it should not adversely affect the performance or behavior of the 
application execution. In order to minimize the overhead of exporting trace 
events we are planning on using the relatively new Websocket protocol [5], 
which keeps open a bidirectional communications channel over the HTTP port. 
By keeping open a long lived websocket connection and sending size efficient 
messages, we do not suffer the TCP and HTTP connection establishment overhead 
for each trace event. Events can also be compressed and buffered for when a 
network connection is again available.  We will use a websocket enabled HTTP 
server, and establish feasibility of running the server on a hosted cloud 
provider, or on the LAN if performance dictates. If the cloud access penalty is 
acceptable, we believe this could be a superior use model that would make people
more likely to use the technology.

If time permits, we would like to build an automated instrumentation insertion 
script, which would detect the appropriate spots in the App code to insert our 
code. This would make it easy for any developer to apply this  LRT capability to
an existing app, just by running the script and making a debug build of their 
app. Code inserted into the App will communicate with a background service on 
the phone, which will then buffer and export the messages to the server.



[1] Web Design Anti-patterns http://webdesign.tutsplus.com/articles/user-experience-articles/the-world-of-ux-anti-patterns/?utm_content=buffer5f64b&utm_source=buffer&utm_medium=linkedin&utm_campaign=Buffer

[2] NSLogger - Remote logging for iOS https://github.com/fpillet/NSLogger

[3] Cocoa Lumberjack - remote logging for iOS https://github.com/robbiehanson/CocoaLumberjack

[4] microlog4android - https://code.google.com/p/microlog4android/

[5] Websocket Protocol - http://tools.ietf.org/html/rfc6455

[6] Android remote debugging tool https://developers.google.com/chrome-developer-tools/docs/remote-debugging


\bibliography{citations}

\end{document}

